\begin{center}

\textbf{МИНОБРНАУКИ РОССИИ}

\vspace{1em}
Федеральное образовательное учреждение\\
высшего профессионального образования\\
«Санкт-Петербургский политехнический университет Петра Великого»

\end{center}

\vspace{1em}

\begin{multicols}{2}

    \phantom{foo}

    \columnbreak

    \center{УТВЕРЖДЕН}

    \begin{flushleft}
        Протоколом заседания выпускающей кафедры
        \uline{<<Компьютерные системы и программные технологии>> \hfill}

        \vspace{1em}

        от <<\uline{\hphantom{200}}>> \uline{\hphantom{декабрььььь}}
        20\uline{\hphantom{130}} г. № \uline{\hphantom{1300}}
    \end{flushleft}
\end{multicols}

\vspace{15pt}

\begin{center}
    \textbf{ИНДИВИДУАЛЬНЫЙ ПЛАН РАБОТЫ СТУДЕНТА, ОБУЧАЮЩЕГОСЯ ПО ПРОГРАММЕ
    МАГИСТЕРСКОЙ ПОДГОТОВКИ}

    \vspace{1em}

    Скрипаль Борис Алексеевич\hrule
\end{center}

\begin{flushleft}
    Кафедра \uline{\textit{Компьютерных систем и программных технологий} \hfill}

    \uline{\hfill}

    Форма обучения \uline{\textit{очная} \hfill}

    Направление подготовки \uline{\textit{Информатика и вычислительная техника} \hfill}

    \uline{\hfill}

    Магистерская программа \textit{\uline{Технологии проектирования системного и прикладного программного обеспечения\hfill}}

    \vspace{1em}

    Руководитель магистерской программы \uline{\hfill}

    \uline{\hfill}

    Научный руководитель \uline{\textit{Ицыксон Владимир Михайлович, к.т.н., доцент} \hfill}

    Период обучения в магистратуре \uline{\textit{сентябрь 2015 - июнь 2017} 
    \hfill}

    Тема магистерской диссертации \textit{\uline{Исследование и разработка 
    аспектно-ориентированных расширений языка Kotlin \hfill}}

    \vspace{1em}

    утверждена на заседании кафедры <<\uline{\hphantom{200}}>> 
    \uline{\hphantom{ноябрь}}
    16\uline{\hphantom{130}} г., протокол № \uline{\hphantom{1300}}

    \vspace{1em}

    Заведующий кафедрой \uline{\hfill} В.М. Ицыксон

    \vspace{1em}

    Студент \uline{\hfill} Б.А. Скрипаль
\end{flushleft}

\documentclass{beamer}
\usepackage[english,russian]{babel}
\usepackage[utf8]{inputenc}
\usepackage{listings}
\usepackage{xcolor}

\lstdefinestyle{base}{
  language=Java,
  emptylines=1,
  breaklines=true,
  basicstyle=\ttfamily\color{black},
  moredelim=**[is][\color{red}]{@}{@},
  showstringspaces=false,
}

% Стиль презентации
\usetheme{Warsaw}
\begin{document}
	
	\setbeamertemplate{navigation symbols}{}
	\setbeamertemplate{footline}{%
		\hspace{0.94\paperwidth}%
		\usebeamerfont{title in head/foot}%
		\insertframenumber\,/\,\inserttotalframenumber%
	}
	\title[Разработка АОП для Kotlin]
	{Исследование и разработка аспектно-ориентированных расширений языка 
	Kotlin}
	
	\author[Б.А. Скрипаль]{
		Скрипаль Б.А. гр. 63501/3\\
		Руководитель: Ицыксон В.М.\\
		Аттестация №3
	}
	\institute{Санкт-Петербургский политехнический университет Петра Великого}
	\date[15.02.2017]{}  
	% Создание заглавной страницы
	\frame{\titlepage} 
	
	\begin{frame}
		\frametitle{План работы}
		\begin{itemize}
			\item [\checkmark] Обзор существующих аспектно-ориентированных 
			расширений для других языков;
			\item [\checkmark] Формулирование требований к создаваемому 
			расширению;
			\item [\checkmark] Начальное описание грамматики аспектов для языка 
			Kotlin;
			\item [+] Анализ PSI и построение срезов;
			\item [+] Применение советов к программе;
			\item [+] Доработка грамматики аспектов;
			\item [--] Реализация плагина для IntelliJ IDEA;
			\item [--] Тестирование;
			\item [--] Написание пояснительной записки.
		\end{itemize}
	\end{frame}
	
	\begin{frame}
		\frametitle{Состояние дел}
        Сделано:
		\begin{itemize}
            \item Анализ промежуточного представления программы;
            \item Создание прототипа;
            \item Начато написание пояснительной записки.
        \end{itemize}

        Планируется сделать:
        \begin{itemize}
            \item Доработка способа внедрения советов;
            \item Доработка формирования срезов;
            \item Доработка прототипа;
            \item Продолжение написания пояснительной записки.
        \end{itemize}
        Оценка степени готовности:
        \begin{itemize}
            \item Практическая часть - 30\%;
            \item Пояснительная записка - 10\%.
        \end{itemize}
    \end{frame}

    \begin{frame}[fragile=singleslide]
    	\frametitle{Немного о том, что сделано}
    	\begin{itemize}
    		\item Описание срезов:
    			\begin{lstlisting}[frame=single,style=base]
pointcut fooPC(): execution(fun Foo.foo())
pointcut printPC(): call(fun kotlin.io.println(String))
    			\end{lstlisting}
    		\item Описание советов:
    			\begin{lstlisting}[frame=single,style=base]
after(): fooPC() && printPC() {
  println("Hello after!!")
}

before(): fooPC() && printPC() {
  println("Hello before!!")
}
    			\end{lstlisting}
    		\item Разметка элементов PSI.
    	\end{itemize}
    \end{frame}

    \begin{frame}[fragile=singleslide]
    	\frametitle{Немного о том, что сделано}
    	При применении кода совета к точке включения, они оборачиваются в
    	функцию <<run>>;
    	\begin{itemize}
    		\item[+] Легко вставлять совет, например, когда функции вызываются
    		  последовательно, например:
    		  \begin{lstlisting}[frame=single,style=base]
val a = a.foo().bar()
    		  \end{lstlisting}
    		  превращается в:
    		  \begin{lstlisting}[frame=single,style=base]
val a = a.run{ val buf = foo(); advice_code; buf}.bar()
    		  \end{lstlisting}
    		\item [--] При применении нескольких советов в одном месте к одной
              точке включения нельзя предсказать порядок выполнения советов;
    		\item [--] При внесении точки включения внутрь тела функции <<run>>
    		  необходимо обновлять метки.
    	\end{itemize}
    \end{frame}

    \begin{frame}[fragile=singleslide]
    	\frametitle{Немного о том, что сделано}
    		\begin{lstlisting}[frame=single,style=base]
class Foo {
  fun foo() {
@    run{@
@      val ____a = run{@
@        println("Hello before!!")@
        println("Hello world")
@      }@
@      println("Hello after!!")@
@      ____a@
@    }@
    this.bar(1,this)
  }

  fun bar(a: Int, b: Foo) {
    println("Bar hello")
  }
}
    	\end{lstlisting}
    \end{frame}
\end{document}
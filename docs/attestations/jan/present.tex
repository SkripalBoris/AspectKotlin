\documentclass{beamer}
\usepackage[english,russian]{babel}
\usepackage[utf8]{inputenc}
\usepackage{listings}
% Стиль презентации
\usetheme{Warsaw}
\begin{document}
	
	\setbeamertemplate{navigation symbols}{}
	\setbeamertemplate{footline}{%
		\hspace{0.94\paperwidth}%
		\usebeamerfont{title in head/foot}%
		\insertframenumber\,/\,\inserttotalframenumber%
	}
	\title[Разработка АОП для Kotlin]
	{Исследование и разработка аспектно-ориентированных расширений языка 
	Kotlin}
	
	\author[Б.А. Скрипаль]{
		Б.А. Скрипаль гр. 63501/3\\
		Руководитель: Ицыксон В.М.\\
		Аттестация №3
	}
	\institute{Санкт-Петербургский политехнический университет Петра Великого}
	\date[15.02.2017]{}  
	% Создание заглавной страницы
	\frame{\titlepage} 
	
	\begin{frame}
		\frametitle{План работы}
		\begin{itemize}
			\item [\checkmark] Обзор существующих аспектно-ориентированных 
			расширений для других языков;
			\item [\checkmark] Формулирование требований к создаваемому 
			расширению;
			\item [\checkmark] Начальное описание грамматики аспектов для языка 
			Kotlin;
			\item [\checkmark --] Анализ PSI и построение срезов;
			\item [\checkmark --] Применение советов к программе;
			\item [\checkmark --] Доработка грамматики аспектов;
			\item [--] Реализация плагина для IntelliJ IDEA;
			\item [--] Тестирование;
			\item [--] Написание пояснительной записки.
		\end{itemize}
	\end{frame}
	
	\begin{frame}
		\frametitle{Состояние дел}
        Сделано:
		\begin{itemize}
            \item Анализ промежуточного представления программы;
            \item Создание прототипа;
            \item Начато написание пояснительной записки.
        \end{itemize}

        Планируется сделать:
        \begin{itemize}
            \item Доработка способа внедрения советов;
            \item Доработка формирования срезов;
            \item Доработка прототипа;
            \item Продолжение написания пояснительной записки.
        \end{itemize}
        Оценка степени готовности:
        \begin{itemize}
            \item Практическая часть - 30\%;
            \item Пояснительная записка - 10\%.
        \end{itemize}
    \end{frame}
\end{document}
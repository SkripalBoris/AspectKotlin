\documentclass{beamer}
\usepackage[english,russian]{babel}
\usepackage[utf8]{inputenc}
\usepackage{listings}
% Стиль презентации
\usetheme{Warsaw}
\begin{document}
	
	\setbeamertemplate{navigation symbols}{}
	\setbeamertemplate{footline}{%
		\hspace{0.94\paperwidth}%
		\usebeamerfont{title in head/foot}%
		\insertframenumber\,/\,\inserttotalframenumber%
	}
	\title[Разработка АОП для Kotlin]
	{Исследование и разработка аспектно - ориентированных расширений языка 
	Kotlin}
	
	\author[Б.А. Скрипаль]{
		Б.А. Скрипаль гр. 63501/3\\
		Руководитель: Ицыксон В.М.\\
		Аттестация №2
	}
	\institute{Санкт-Петербургский политехнический университет Петра Великого}
	\date[16.11.2016]{}  
	% Создание заглавной страницы
	\frame{\titlepage} 
	
	\begin{frame}
		\frametitle{Описание задачи}
		Аспектно - ориентированный подход позволяет описывать и внедрять 
		сквозную функциональность.
		
		Задача: разработка аспектно - ориентированного расширения для языка 
		Kotlin.
	\end{frame}
	
	\begin{frame}
		\frametitle{План работы}
		\begin{itemize}
			\item [\checkmark] обзор существующих аспектно - ориентированных 
			расширений для других языков;
			\item [\checkmark] формулирование требований к создаваемому 
			расширению;
			\item [\checkmark] начальное описание грамматики аспектов для языка 
			Kotlin;
			\item Анализ AST и построение срезов;
			\item Применение советов к программе;
			\item Доработка грамматики аспектов;
			\item Реализация плагина для IntelliJ IDEA;
			\item Тестирование;
			\item Написание пояснительной записки.
		\end{itemize}
	\end{frame}
	
	\begin{frame}
		\frametitle{Что сделано}
		Произведен обзор существующий аспектно - ориентированных расширений:
		\begin{itemize}
			\item Spring AOP;
			\item AspectJ;
			\item PostShark
		\end{itemize}
		
		Описана начальная грамматика аспектов при помощи ANTLR4.
	\end{frame}
\end{document}
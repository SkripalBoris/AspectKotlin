\documentclass{beamer}
\usepackage[english,russian]{babel}
\usepackage[utf8]{inputenc}
\usepackage{listings}
\usepackage{xcolor}

\lstdefinestyle{base}{
  language=Java,
  emptylines=1,
  breaklines=true,
  basicstyle=\ttfamily\color{black},
  moredelim=**[is][\color{red}]{@}{@},
  showstringspaces=false,
}

% Стиль презентации
\usetheme{Warsaw}
\begin{document}
	
	\setbeamertemplate{navigation symbols}{}
	\setbeamertemplate{footline}{%
		\hspace{0.94\paperwidth}%
		\usebeamerfont{title in head/foot}%
		\insertframenumber\,/\,\inserttotalframenumber%
	}
	\title[Разработка АОП для Kotlin]
	{Разработка аспектно-ориентированного расширения для языка Kotlin}
	
	\author[Б.А. Скрипаль]{
		Скрипаль Б.А. гр. 63501/3\\
		Руководитель: Ицыксон В.М.\\
		Аттестация №4
	}
	\institute{Санкт-Петербургский политехнический университет Петра Великого}
	\date[25.04.2017]{25.04.2017}  
	% Создание заглавной страницы
	\frame{\titlepage} 
	
	\begin{frame}
		\frametitle{План работы}
		\begin{itemize}
			\item [\checkmark] Обзор существующих аспектно-ориентированных 
			расширений для других языков
			\item [\checkmark] Формулирование требований к создаваемому 
			расширению
			\item [\checkmark] Начальное описание грамматики аспектов для языка 
			Kotlin
			\item [\checkmark] Анализ PSI и построение срезов
			\item [\checkmark] Применение советов к программе
			\item [+] Доработка грамматики аспектов
			\item [+] Тестирование
			\item [+] Написание пояснительной записки
			\item [--] Реализация плагина для IntelliJ IDEA
		\end{itemize}
	\end{frame}
	
	\begin{frame}
		\frametitle{Состояние дел}
        Сделано:
		\begin{itemize}
            \item Возможность задания в срезах inline и extension функций
            \item Способы внедрения совета
            \item Написана статья
            \item Пишется пояснительная записка
        \end{itemize}

        Планируется сделать:
        \begin{itemize}
            \item Обращение к аргументам функций
            \item Доработка способов формирования срезов
            \item Доработка прототипа
            \item Продолжение написания пояснительной записки
        \end{itemize}
        Оценка степени готовности:
        \begin{itemize}
            \item Практическая часть - 60\%;
            \item Пояснительная записка - 40\%.
        \end{itemize}
    \end{frame}
\end{document}
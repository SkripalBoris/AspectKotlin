%%%%%%%%%%%%%%%%%%%%%%%%%%%%%%%%%%%%%%%%%%%%%%%%%%%%%%%%%%%%%%%%%%%%%%%%%%%%%%%%
\chapter{Тестирование прототипа аспектно-ориентированного расширения}
%%%%%%%%%%%%%%%%%%%%%%%%%%%%%%%%%%%%%%%%%%%%%%%%%%%%%%%%%%%%%%%%%%%%%%%%%%%%%%%%
В данном разделе ставится задача проведения тестирования разработанного прототипа на предмет наличия дефектов и соответствия поставленным функциональным и нефункциональным требованиям.
%%%%%%%%%%%%%%%%%%%%%%%%%%%%%%%%%%%%%%%%%%%%%%%%%%%%%%%%%%%%%%%%%%%%%%%%%%%%%%%%
\section{Программа испытаний}
\label{sec:testing_program}
%%%%%%%%%%%%%%%%%%%%%%%%%%%%%%%%%%%%%%%%%%%%%%%%%%%%%%%%%%%%%%%%%%%%%%%%%%%%%%%%
Для того, чтобы убедиться в работоспособности разработанного прототипа необходимо проверить его корректность несколькими способами, а именно:
\begin{itemize}
	\item проверка корректности построения модели аспектов;
	\item проверка правильности построения срезов;
	\item проверка внедрения кода советов в PSI целевой программы.
\end{itemize}

На начальном этапе проверку корректности грамматики можно при помощи графических средств, предоставляемых плагином ANTLR для среды разработки IntelliJ IDEA.
На более поздних этапах проверку корректности разбора описания аспектов и построения среза можно осуществить при помощи функционального тестирования.\

Для проверки корректности построения срезов и применения кода советов необходим анализ непосредственно PSI.
Из-за того, что сама по себе проверка PSI является сложной задачей, было решено
проверить корректность работы на уровне функций следующим образом:
\begin{enumerate}
	\item составить простую целевую программу, к которой будут применяться аспекты;
    \item составить набор аспектов, которые будут применяться к целевой
          программе;
    \item составить набор эталонных файлов, представляющих из себя исходные
          файлы целевой программы, к которым вручную применены описанные выше
          аспекты;
	\item модифицировать PSI целевой программы, а именно произвести составление
	      модели аспектов, разметку PSI и применить их к PSI проекта;
	\item сформировать на основе  модифицированных виртуальные файлов исходные
	      файлы на языке Kotlin;
	\item сравнить на уровне токенов полученные файлы с эталонными исходными
	      файлами, в которых аспекты применены вручную.		  
\end{enumerate}

После успешной проверки корректности работы на уровне функций, необходимо
убедиться в том, что после применения аспектов программа остается корректной
и работает согласно нашим ожиданиям.
Это можно сделать, например, поместив в код совета вывод отладочных сообщений,
и после выполнения программы необходимо сравнить результат работы программы с 
ожидаемым.
%%%%%%%%%%%%%%%%%%%%%%%%%%%%%%%%%%%%%%%%%%%%%%%%%%%%%%%%%%%%%%%%%%%%%%%%%%%%%%%%
\section{Методика испытаний}
\label{sec:testing_methodology}
%%%%%%%%%%%%%%%%%%%%%%%%%%%%%%%%%%%%%%%%%%%%%%%%%%%%%%%%%%%%%%%%%%%%%%%%%%%%%%%%
%%%%%%%%%%%%%%%%%%%%%%%%%%%%%%%%%%%%%%%%%%%%%%%%%%%%%%%%%%%%%%%%%%%%%%%%%%%%%%%%
\section{Проведение испытаний}
\label{sec:testing_run}
%%%%%%%%%%%%%%%%%%%%%%%%%%%%%%%%%%%%%%%%%%%%%%%%%%%%%%%%%%%%%%%%%%%%%%%%%%%%%%%%
%%%%%%%%%%%%%%%%%%%%%%%%%%%%%%%%%%%%%%%%%%%%%%%%%%%%%%%%%%%%%%%%%%%%%%%%%%%%%%%%
\section{Выводы}
\label{sec:ch5_conclusion}
%%%%%%%%%%%%%%%%%%%%%%%%%%%%%%%%%%%%%%%%%%%%%%%%%%%%%%%%%%%%%%%%%%%%%%%%%%%%%%%%
%%%%%%%%%%%%%%%%%%%%%%%%%%%%%%%%%%%%%%%%%%%%%%%%%%%%%%%%%%%%%%%%%%%%%%%%%%%%%%%%

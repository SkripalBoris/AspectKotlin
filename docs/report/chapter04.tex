%%%%%%%%%%%%%%%%%%%%%%%%%%%%%%%%%%%%%%%%%%%%%%%%%%%%%%%%%%%%%%%%%%%%%%%%%%%%%%%%
\chapter{Разработка прототипа аспектно-ориентированного расширения}
%%%%%%%%%%%%%%%%%%%%%%%%%%%%%%%%%%%%%%%%%%%%%%%%%%%%%%%%%%%%%%%%%%%%%%%%%%%%%%%%
В данном разделе рассматривается реализация прототипа в соответствии с
поставленными требованиями и разработанной архитектурой.
%%%%%%%%%%%%%%%%%%%%%%%%%%%%%%%%%%%%%%%%%%%%%%%%%%%%%%%%%%%%%%%%%%%%%%%%%%%%%%%%
\section{Средства разработки}
\label{sec:dev_tools}
%%%%%%%%%%%%%%%%%%%%%%%%%%%%%%%%%%%%%%%%%%%%%%%%%%%%%%%%%%%%%%%%%%%%%%%%%%%%%%%%
Для разработки синтаксиса описания аспектов было решено использовать язык ANTLR4
про следующим причинам:
\begin{itemize}
	\item Данный язык позволяет достаточно компактно описывать грамматики и
		  генерировать парсеры по составленным грамматикам.
	\item Возможность генерации парсеров на языке Java.
	\item Существование грамматики описания аспектов на данном языке для
		  фреймворка AspectJ, синтаксис которого было решено взять за основу
		  разрабатываемой грамматики.
\end{itemize}

При разработке прототипа было решено использовать язык Kotlin.
Данный выбор был обусловлен следующими причинами:
\begin{itemize}
	\item Полная совместимость с языком Java и, как следствие, возможность
		  использовать библиотеки, разработанные для языка Java.
	\item Необходимость обеспечить достаточно высокую производительность, т.к.
		  целевые программы могут иметь достаточно большие размеры.
	\item Наличие опыта разработки на данном языке у автора работы, что
		  позволит уменьшить сроки разработки прототипа.
\end{itemize}
Разработка среды велась на языке Kotlin версии 1.0.6.
В качестве целевых программ также подразумеваются программы, написанные на языке Kotlin версии 1.0.6.

Сборка проекта производится при помощи фреймворка Apache Maven.
Данный фреймворк позволяет декларативным образом описывать проект сборки, а
также подключать сторонние библиотеки.
Одной из отличительных особенностей Maven является наличие центрального
репозитория, содержащего большое колличество библиотек, как для языка Java, так
и для языка Kotlin.
Вся информация о сборке (описание циклов сборки, подключаемые библиотеки и т.д.)
описывается в файле pom.xml (Project Object Model) в формате XML.
Еще одним критерием, по которому выбор был сделан в пользу Maven является наличие плагина, позволяющего компилировать программы на Kotlin.
%%%%%%%%%%%%%%%%%%%%%%%%%%%%%%%%%%%%%%%%%%%%%%%%%%%%%%%%%%%%%%%%%%%%%%%%%%%%%%%%
\section{Структура проекта}
\label{sec:project_structure}
%%%%%%%%%%%%%%%%%%%%%%%%%%%%%%%%%%%%%%%%%%%%%%%%%%%%%%%%%%%%%%%%%%%%%%%%%%%%%%%%
Условно, проект можно разбить на две логические части: часть описания
синтаксиса аспектов и часть, реализующая внедрение аспектов в целевую программу.
Ниже будет приведено описание процесса разработки двух вышеописанных частей.
%%%%%%%%%%%%%%%%%%%%%%%%%%%%%%%%%%%%%%%%%%%%%%%%%%%%%%%%%%%%%%%%%%%%%%%%%%%%%%%%
\subsection{Разработка грамматики аспектов}
\label{sub:aspect_grammar_part_development}
%%%%%%%%%%%%%%%%%%%%%%%%%%%%%%%%%%%%%%%%%%%%%%%%%%%%%%%%%%%%%%%%%%%%%%%%%%%%%%%%
Для удобства, описание грамматики аспектов было разделено между двумя файлами:
\textit{AspectGrammar.g4} (исходный код приведен в листинге
\ref{listings:AspectGrammar}) и \textit{KotlinGrammar.g4} (исходный код
приведен в листинге \ref{listings:KotlinGrammar}).
В файле \textit{AspectGrammar.g4} описана чать грамматики, отвечающая за
грамматику непосредственно аспектов, как например, правила описания аспектов,
срезов, советов и т.д.
В файле \textit{KotlinGrammar.g4} содержится описание грамматики, относящейся к
языку Kotlin, например, стандартные типы языка, правила описания параметров
методов, модификаторы методов и т.д.
%%%%%%%%%%%%%%%%%%%%%%%%%%%%%%%%%%%%%%%%%%%%%%%%%%%%%%%%%%%%%%%%%%%%%%%%%%%%%%%%
\subsection{Разработка части, отвечающей за внедрение аспектов}
\label{sub:aspect_weaving_part_development}
%%%%%%%%%%%%%%%%%%%%%%%%%%%%%%%%%%%%%%%%%%%%%%%%%%%%%%%%%%%%%%%%%%%%%%%%%%%%%%%%
%%%%%%%%%%%%%%%%%%%%%%%%%%%%%%%%%%%%%%%%%%%%%%%%%%%%%%%%%%%%%%%%%%%%%%%%%%%%%%%%

%%%%%%%%%%%%%%%%%%%%%%%%%%%%%%%%%%%%%%%%%%%%%%%%%%%%%%%%%%%%%%%%%%%%%%%%%%%%%%%%
\intro
%%%%%%%%%%%%%%%%%%%%%%%%%%%%%%%%%%%%%%%%%%%%%%%%%%%%%%%%%%%%%%%%%%%%%%%%%%%%%%%%
\nomenclature{АОП}{Аспектно-Ориентированное Программирование}
\nomenclature{ООП}{Объектно-Ориентированное Программирование}

С развитием языков программирования появляется все больше парадигм
программирования, представляющих из себя совокупность идей и понятий,
определяющих стиль написания программ.
Первые программы писались в машинных кодах и на языках ассемблера, впоследствии,
с ростом уровня сложности решаемых задач появились языки высокого уровня,
которые позволили уменьшить усилия программиста при написании программного кода
и увеличить производительность.
Одной из первых парадигм, используемых в языках высокого уровня, стала парадигма
\quotes{Процедурного программирования}, однако, со временем, появились и другие
парадигмы программирования, такие как \quotes{Структурное программирование},
\quotes{Автоматное программирование}, \quotes{ООП} и др.

АОП так же является одной из парадигм программирования, которая появилась как
инструмент для решения проблем, с которыми, существующие на момент появления
АОП, справлялись недостаточно хорошо.
Если быть точнее, то основной причиной появления парадигмы АОП является
сложность внедрения сквозной функциональности~\cite{crosscutting_conserns} в
код, написанный при помощи объектно-ориентированного подхода.

В данной работе будет рассмотрен аспектно-ориентированная парадигма
программирования, существующие аспектно-ориентированные расширения для
различных объектно-ориентированных языков программирования, а так же описание
разработки аспектно-ориентированного расширения для языка Kotlin.
Kotlin --- статически типизированный язык программирования, работающий поверх
JVM и разрабатываемый компанией JetBrains. Данный язык компилируется в байт-код и полностью совместим с языком Java.
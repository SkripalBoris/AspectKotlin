%%%%%%%%%%%%%%%%%%%%%%%%%%%%%%%%%%%%%%%%%%%%%%%%%%%%%%%%%%%%%%%%%%%%%%%%%%%%%%%%
\intro
%%%%%%%%%%%%%%%%%%%%%%%%%%%%%%%%%%%%%%%%%%%%%%%%%%%%%%%%%%%%%%%%%%%%%%%%%%%%%%%%
\nomenclature{АОП}{Аспектно-Ориентированное Программирование}
\nomenclature{ООП}{Объектно-Ориентированное Программирование}

В данной работе рассматривается подход, позволяющий использовать аспектно-ориентированную парадигму программирования при написании программ на языке Kotlin.

С развитием языков программирования появлялось все больше парадигм, представляющих из себя совокупность идей и понятий, определяющих стиль написания программ.
Одной из таких парадигм является объектно-ориентированное программирование (ОПП).
Одной из идей данного подхода является локализация данных и функциональности, оперирующей с этими данными.
Со временем было замечено, что часть функциональности не сосредоточена в одном месте, а, наоборот, распределена по всей программе.
Данную функциональность стали называть \quotes{сквозная функциональность}, тем самым подчеркивая её рассредоточенность.
Парадигма ООП не позволяла в ряде случаев компактно описывать данную функциональность, что привело к появлению новой парадигмы, дополняющей объектно-ориентированный подход.

Аспектно-ориентированный подход (АОП) был предложен как решение проблемы
описания сквозной функциональности в объектно-ориентированных программах.
Впервые подход был представлен в 1997 году Грегором Кичалесом в работе
\quotes{Aspect-oriented programming}~\cite{kiczales_aop}.
В предложенном подходе сквозная функциональность описывается отдельно от
объектно-ориентированного кода программы и внедряется на этапе компиляции.
Такое разделение позволяет не только компактно описывать сквозную 
функциональность, но и делать её внедрение прозрачным для программиста.

Парадигма АОП предложила элегантное решение для ряда задач, как, например, 
протоколирование и трассировка программ, работа с транзакциями, обработка
ошибок, а также некоторых других.
АОП стал интенсивно развиваться и, в настоящий момент, аспектно-ориентированные
расширения созданы для большинства объектно-ориентированных языков
программирования, как, например, C++~\cite{aspectC_homepage},
C\#~\cite{postsharp_doc}, Java~\cite{aspectj_doc,springAOP_doc},
Python~\cite{spring_python} и т.п.

Язык Kotlin --- молодой мультипарадигменный язык программирования,
разрабатываемый компанией JetBrains.
Основная цель языка Kotlin --- быть компактной, выразительной и надежной заменой
языка Java.
При этом язык обеспечивает полную совместимость с программами на языке
Java.

Наличие существенного числа новых конструкций не позволяет использовать уже 
готовые АОП-расширения для JVM-совместимых языков, поэтому
аспектно-ориентированное  расширение для языка Kotlin необходимо разрабатывать,
специально ориентируясь на особенности языка Kotlin.
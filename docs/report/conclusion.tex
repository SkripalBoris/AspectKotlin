%%%%%%%%%%%%%%%%%%%%%%%%%%%%%%%%%%%%%%%%%%%%%%%%%%%%%%%%%%%%%%%%%%%%%%%%%%%%%%%%
\conclusion
%%%%%%%%%%%%%%%%%%%%%%%%%%%%%%%%%%%%%%%%%%%%%%%%%%%%%%%%%%%%%%%%%%%%%%%%%%%%%%%%
В результате выполнении диссертации был разработан подход, позволяющий использовать парадигму аспектно-ориентированного программирования при написании программ на языке Kotlin, разработан синтаксис аспектов с учетом особенностей языка и разработан прототип, реализующий предложенный подход.

В ходе работы были рассмотрен ряд существующих на данный момент АОП-расширений различных языков, используемые в них способы описания и внедрения аспектов (раздел~\ref{ch:aop_overview}).
На основе этого обзора было принято решение о целесообразности разработки подхода, позволяющего использовать аспектно-ориентированный подход при разработки программ на языке Kotlin.

На основе анализа существующих расширений и требований, поставленных в
разделе~\ref{ch:tasks}, была предложена грамматика аспектов, архитектура инструментальной среды, которая была разделена на три фрагмента --- грамматика и парсеры аспектов, компилятор Kotlin и часть, отвечающая за внедрение аспектов в PSI программы (раздел~\ref{ch:extension_design}).
Реализация прототипа на основе предложенной архитектуры описана в разделе ~\ref{ch:develop}.

На заключительном этапе разработки (раздел~\ref{ch:testing}) было проведено
функциональное тестирование среды, показавшее соответствие разработанного прототипа поставленным требованиям.

Разработанный подход можно применять для внедрения сквозной функциональности в программы на языке Kotlin.
Построение срезов, к которым будет применена сквозная функциональность может производиться с использованием сигнатур \textit{call}, \textit{execution} и \textit{target} с учетом специфичных для языка Kotlin конструкций (inline и extension методы и т.д.).
Также при описании срезов можно использовать логические операции дизъюнкции, конъюнкции и инверсии.
Сквозная функциональность может быть применена до, после и вместо точки внедрения.

Дальнейшим развитием проекта является поддержка новых сигнатур и способов внедрения сквозной функциональности.
Также планируется ввести возможность создания вспомогательных методов и переменных внутри аспекта.
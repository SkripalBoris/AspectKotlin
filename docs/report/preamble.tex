%%%%%%%%%%%%%%%%%%%%%%%%%%%%%%%%%%%%%%%%%%%%%%%%%%%%%%%%%%%%%%%%%%%%%%%%%%%%%%%%
%%
%% Преамбула
%%
%%%%%%%%%%%%%%%%%%%%%%%%%%%%%%%%%%%%%%%%%%%%%%%%%%%%%%%%%%%%%%%%%%%%%%%%%%%%%%%%
\documentclass[%
  a5paper,
  subf,
  href,
  master,
  dotsinheaders
]{csse-fcs}

\usepackage[T2A]{fontenc}
\usepackage[utf8]{inputenc}
\usepackage[english,russian]{babel}
\usepackage[figuresright]{rotating}
\usepackage{tabularx}
\usepackage{float}
 
% Компактные списки
\usepackage{mdwlist}
\makecompactlist{mdwitemize}{itemize}
\makecompactlist{mdwenumerate}{enumerate}
\makecompactlist{mdwdescription}{description}

% Таблицы
\usepackage{array}

% Листинги
\usepackage{listings}

% TODOs
\usepackage[%
  colorinlistoftodos,
  shadow
]{todonotes}

% Путь к каталогу со всеми рисунками
\graphicspath{{fig/}}

% Автоконвертер для EPS
\usepackage{epstopdf}

% Генератор текста
\usepackage{blindtext}

\renewcommand{\theenumi}{\arabic{enumi}}
\renewcommand{\labelenumi}{\theenumi.}
\renewcommand{\theenumii}{\asbuk{enumii}}
\renewcommand{\labelenumii}{\theenumii.}
\renewcommand{\theenumiii}{\roman{enumiii}.}
\renewcommand{\labelenumiii}{\theenumiii}
\renewcommand{\theenumiv}{\Alph{enumiv}.}
\renewcommand{\labelenumiv}{\theenumiv}
\makeatletter
\renewcommand\p@enumii{\theenumi.}
\renewcommand\p@enumiii{\theenumi.\theenumii.}
\makeatother

%%%%%%%%%%%%%%%%%%%%%%%%%%%%%%%%%%%%%%%%%%%%%%%%%%%%%%%%%%%%%%%%%%%%%%%%%%%%%%%%

%%%%%%%%%%%%%%%%%%%%%%%%%%%%%%%%%%%%%%%%%%%%%%%%%%%%%%%%%%%%%%%%%%%%%%%%%%%%%%%%
\chapter{Постановка задачи}
\label{ch:tasks}
%%%%%%%%%%%%%%%%%%%%%%%%%%%%%%%%%%%%%%%%%%%%%%%%%%%%%%%%%%%%%%%%%%%%%%%%%%%%%%%%
Как было сказано в подразделе~\ref{sec:aop_extension_overwiev}, существующие на данный момент АОП-расширения или не подходят для их использования при написании программ на языке Kotlin, или не могут учитывать некоторые специфичные языковые конструкции при формировании срезов.
Также, единственное АОП-расширение, официально поддерживающее язык Kotlin, осуществляет только динамическое внедрение аспектов, что может быть неудобным в некоторых ситуациях.
%%%%%%%%%%%%%%%%%%%%%%%%%%%%%%%%%%%%%%%%%%%%%%%%%%%%%%%%%%%%%%%%%%%%%%%%%%%%%%%%
\section{Функциональные требования}
%%%%%%%%%%%%%%%%%%%%%%%%%%%%%%%%%%%%%%%%%%%%%%%%%%%%%%%%%%%%%%%%%%%%%%%%%%%%%%%%
Таким образом, исходя из описанных проблем, ставится задача разработки аспектно-ориентированного расширения языка Kotlin.
Разрабатываемое расширение должно удовлетворять следующим требованиям:
\begin{itemize}
    \item Грамматика описания аспектов должна быть стилистически схожа с грамматикой языка Kotlin и быть описана отдельно.
    % Переформулировать и в конец
    \item Поддержка следующих способов вставки советов:
    \begin{itemize}
        \item до точки объединения;
        \item после точки объединения;
        \item вместо точки объединения.
    \end{itemize}
    \item Поддержка следующих элементов работы со срезами:
    \begin{itemize}
        \item Возможность использования метасимвола \quotes{*} (замена любой строки символов) при задании имени классов, пакетов, методов;
        \item Возможность задания модификаторов методов, например, \textit{public}, \textit{private}, \textit{protected}, \textit{inline} и т.д.;
        \item Возможность выделения extension методов при формировании срезов;
        \item Возможность указывать nullability модификаторы параметров при описании методов;
        \item Выделение всех мест вызова метода при помощи сигнатуры \textit{call};
        \item Выделение всех мест исполнения метода при помощи сигнатуры \textit{execution};
        \item Выделение всех мест, в котором производится вызов метода на элементе заданного класса при помощи сигнатуры \textit{target};
        \item Возможность обращение к элементу на котором был произведен вызов метода внутри кода совета;
        \item Использование операций дизъюнкции, конъюнкции и инверсии при описании срезов.
    \end{itemize}
    \item Возможность использование одних срезов при описании других.
    \item Использование статического способа внедрения советов.
\end{itemize}
%%%%%%%%%%%%%%%%%%%%%%%%%%%%%%%%%%%%%%%%%%%%%%%%%%%%%%%%%%%%%%%%%%%%%%%%%%%%%%%%
\section{Решаемые задачи}
\label{sec:tasks}
%%%%%%%%%%%%%%%%%%%%%%%%%%%%%%%%%%%%%%%%%%%%%%%%%%%%%%%%%%%%%%%%%%%%%%%%%%%%%%%%
Как видно из обзора, представленного в подразделе~\ref{sec:aop_extension_overwiev}, ни одно из существующих на данный момент АОП-расширений не отвечает поставленным требованиям, поэтому было принято решение о разработке собственного АОП-расширения языка Kotlin.
При этом предлагается решить следующие задачи:
\begin{itemize}
    \item проектирование грамматики аспектов;
    \item разработка архитектуры расширения;
    \item разработка модели аспектов;
    \item разработка парсера аспектов;
    \item разработка модуля формирования срезов;
    \item разработка модуля внедрения аспектов;
    \item тестирование разработанного прототипа.
\end{itemize}

В результате успешного выполнения данных задач, необходимо получить расширение, позволяющее использовать аспектно-проектированный подход при написании программ на языке Kotlin.
%%%%%%%%%%%%%%%%%%%%%%%%%%%%%%%%%%%%%%%%%%%%%%%%%%%%%%%%%%%%%%%%%%%%%%%%%%%%%%%%
\section{Выводы}
%%%%%%%%%%%%%%%%%%%%%%%%%%%%%%%%%%%%%%%%%%%%%%%%%%%%%%%%%%%%%%%%%%%%%%%%%%%%%%%%
В данном разделе были сформулированы требования к разрабатываемому расширению, а также поставлены задачи, которые необходимо решить для достижения цели диссертации --- разработки АОП-расширения языка Kotlin.
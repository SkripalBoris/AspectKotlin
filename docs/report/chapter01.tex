%%%%%%%%%%%%%%%%%%%%%%%%%%%%%%%%%%%%%%%%%%%%%%%%%%%%%%%%%%%%%%%%%%%%%%%%%%%%%%%%
\chapter{Обзор существующих аспектно-ориентированных расширений}
%%%%%%%%%%%%%%%%%%%%%%%%%%%%%%%%%%%%%%%%%%%%%%%%%%%%%%%%%%%%%%%%%%%%%%%%%%%%%%%%

%%%%%%%%%%%%%%%%%%%%%%%%%%%%%%%%%%%%%%%%%%%%%%%%%%%%%%%%%%%%%%%%%%%%%%%%%%%%%%%%
\section{Описание аспектно-ориентированного подхода}
%%%%%%%%%%%%%%%%%%%%%%%%%%%%%%%%%%%%%%%%%%%%%%%%%%%%%%%%%%%%%%%%%%%%%%%%%%%%%%%%
Подход, описывающий принципы аспектно-ориентированного программирования, был
предложен группой инженеров исследовательского центра Xerox PARC под
руководством Грегора Кичалеса~\cite{kiczales_aop}.
В данной работе были описаны сложности использования ООП при внедрении сквозной
функциональности в код, а так же было предложено выносить сквозную
функциональность в некоторую отдельную сущность - \quotes{аспект}, для
последующего внедрения в код целевой программы.

%В его работе были описаны парадигма аспектно-ориентированного программирования и
%её основные понятия, способы внедрения аспектов, а так же приведен пример
%расширения языка Java при помощи аспектов.
Позднее, данное подразделение разработало аспектно-ориентированное расширения
для языка Java - AspectJ~\cite{kiczales_aspectj}.
Также в этой работе были более детально описаны основные понятия АОП:
  \begin{itemize}
    \item Аспект - модуль или класс, реализующий сквозную функциональность.
      Аспект изменяет поведение остального кода, применяя совет в точках 
      соединения, определенных некоторым срезом.
    \item Совет - средство оформления кода, которое должно быть вызвано из точки
    соединения.
    Совет может быть выполнен до, после или вместо точки соединения.
	\item Точка соединения - точка в выполняемой программе, где следует
	применить совет.
	Многие реализации АОП позволяют использовать вызовы методов и обращения к
	полям объекта в качестве точек соединения.
	\item Срез - набор точек соединения, определяет, подходит ли данная точка
	соединения к данному совету.
	\item Внедрение - изменение структуры класса и (или) изменение иерархии
	наследования для добавления функциональности аспекта в инородный код.
  \end{itemize}
	
В аспектно-ориентированном программировании может использоваться два способа
внедрения советов: статический и динамический способы.
Статический способ внедрения аспектов подразумевает внедрение советов
непосредственно в программный продукт в процессе построения.
Данный подход является более быстрым, в отличии от динамического подхода,
однако, имеет один недостаток: при изменении аспекта необходимо пересобрать весь
программный продукт.
Такой подход использован, например, в расширении AspectJ.
Динамический подход подразумевает связывание аспектов и исполняемого кода
непосредственно в процессе исполнения программы. Данный подход уступает в
быстродействии статическому связыванию, однако, лишен необходимости пересборки
всего проекта в случае изменения аспектов.

Как было сказано ранее, АОП позволяет выносить сквозную функциональность в
отдельную сущность, что может быть полезно при решении ряда задач, хотя такой
подход и делает код более сложным для понимания, он позволяет компактно
описывать общую функциональность и прозрачно внедрять её, тем самым в ряде
случаев ускоряя и упрощая разработку программного продукта, а так же облегчая
повторное использование кода.

%\begin{figure}[htbp]
%\centering
%\includegraphics[width=\textwidth]{how-to-do-the-actual-research}
%\caption{Рекомендации по проведению исследований в рамках диссертации}%
%\label{fig:how-to-do-research}
%\end{figure}


%%%%%%%%%%%%%%%%%%%%%%%%%%%%%%%%%%%%%%%%%%%%%%%%%%%%%%%%%%%%%%%%%%%%%%%%%%%%%%%%
\section{bar}
%%%%%%%%%%%%%%%%%%%%%%%%%%%%%%%%%%%%%%%%%%%%%%%%%%%%%%%%%%%%%%%%%%%%%%%%%%%%%%%%

\blindtext
It is of great importance that you use correct references in your dissertation.
Resent studies show that it can increase the chances of successful defense
by as much as 3,17 percent~\cite{big,small,russian}.

\Blindtext

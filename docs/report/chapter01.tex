%%%%%%%%%%%%%%%%%%%%%%%%%%%%%%%%%%%%%%%%%%%%%%%%%%%%%%%%%%%%%%%%%%%%%%%%%%%%%%%%
\chapter{Обзор существующих аспектно-ориентированных расширений}
%%%%%%%%%%%%%%%%%%%%%%%%%%%%%%%%%%%%%%%%%%%%%%%%%%%%%%%%%%%%%%%%%%%%%%%%%%%%%%%%

%%%%%%%%%%%%%%%%%%%%%%%%%%%%%%%%%%%%%%%%%%%%%%%%%%%%%%%%%%%%%%%%%%%%%%%%%%%%%%%%
\section{Описание аспектно-ориентированного подхода}
%%%%%%%%%%%%%%%%%%%%%%%%%%%%%%%%%%%%%%%%%%%%%%%%%%%%%%%%%%%%%%%%%%%%%%%%%%%%%%%%
Подход, описывающий принципы аспектно-ориентированного программирования, был
предложен группой инженеров исследовательского центра Xerox PARC под
руководством Грегора Кичалеса~\cite{kiczales_aop}.
В данной работе были описаны сложности использования ООП при внедрении сквозной
функциональности в код, а так же было предложено выносить сквозную
функциональность в некоторую отдельную сущность - \quotes{аспект}, для
последующего внедрения в код целевой программы.
На данном этапе были сформулированы основные идеи развития АОП, а также приведен
пример прототипа языка для описания аспектов.

Позднее, в своей статье \quotes{An Overview of AspectJ}~\cite{kiczales_aspectj},
Грегор ККичалес со своей научной группой представили более детальное и подробное
описание аспектов, а также расширение, позволяющее реализовывать аспекты для
языка Java - AspectJ.
В данной работе были не только было представлено более детальное описание такой
сущности, как \quotes{аспект}, а так же описан синтаксис языка для описания
аспектов и способ внедрения аспектов в код программы.

Под аспектом понимается некоторая сущность, содержащая в себе не только описание
сквозной функциональности, но и правила её добавления в программный код, а
также, возможно, дополнительные функции и (или) переменные.
Сквозная функциональность, внутри аспекта, содержится в сущности, называемой
\quotes{совет} (advice).
Кроме этого, совет содержит описание среза (pointcut), согласно которому
происходит поиск \quotes{точек включения} (join points) в программе и применение
совета согласно некоторым правилам, которые тоже, как правило, описываются
внутри совета.
Для того, чтобы описание срезов было более компактным, а так же чтобы избежать
дублирования когда, правила для среза иногда разбивают на части, при этом каждая
из частей имеет уникальный (в рамках аспекта) идентификатор, который можно
вызывать как при описании советов, так и при описании других срезов.
После этого, на основе срезов, в целевую программу, по некоторым правилам,
внедряется код совета.
Можно выделить следующие способы внедрения кода советов относительно точки
объединения~\footnote{Возможны другие способы внедрения кода советов
относительно точек объединения, однако, данный список способов внедрения
реализуется в большинстве существующих аспектно-ориентированных расширений}:
\begin{itemize}
  \item Before - вставка кода совета перед точкой объединения;
  \item After - вставка кода совета после точки объединения;
  \item Afterreturning - вставка кода совета после возвращения значения;
  \item Afterthrowning - вставка кода совета после возникновения исключения
  (тип исключения, как правило, задается при описании совета);
  \item Around - вставка кода совета до и после точки объединения.
\end{itemize}

Для составления срезов также используется ряд правил поиска, которые могут
комбинироваться между собой при помощи логических операторов \quotes{и},
\quotes{или}, \quotes{не}, а так же вызываться друг другом.
Набор ключевых слов для описания срезов в различных библиотеках отличается, но,
как правило основными примитивами являются:
%%TODO
\begin{itemize}
  \item Выполнение функции;
  \item Вызов функции;
  \item Обработка определенной исключительной ситуации;
  \item Поток управления.
\end{itemize}

%%%%%%%%%%%%%%%%%%%%%%%%%%%%%%%%%%%%%%%%%%%%%%%%%%%%%%%%%%%%%%%%%%%%%%%%%%%%%%%%
\section{Способы внедрения советов}
%%%%%%%%%%%%%%%%%%%%%%%%%%%%%%%%%%%%%%%%%%%%%%%%%%%%%%%%%%%%%%%%%%%%%%%%%%%%%%%%
На данный момент можно выделить два основных способа внедрения сквозной
функциональности в программный продукт: статическое и динамическое связывание~
\cite{static_and_dynamic_AOP}.
Статический подход подразумевает связывание целевой программы и сквозной
функциональности на этапе в процессе построения, в то время как при динамическом
подходе, связывание программного продукта и сквозной функциональности происходит
только в момент исполнения программы.

При использовании статического подхода, внедрение сквозной функциональности
может происходить на одном из трех этапов:
\begin{enumerate}
  \item Перед компиляцией исходных кодов;
  \item Во время процесса компиляции;
  \item Сразу после компиляции исходных кодов.
\end{enumerate}

При динамическом способе внедрения аспектов используются объекты - 
посредники proxy), которые привязываются к объектам, к которым должны быть 
применены аспекты и применяют необходимые советы в процессе выполнения 
программы.
Посредник имеет тот же интерфейс, что и исходный класс, но он не вызывает 
сразу же метод объекта, реализующий класс Callee, есть возможность 
совершить дополнительные действия как до момента вызова (Before Advice), 
так и после него (AfterAdvice)
%В аспектно-ориентированном программировании может использоваться два способа
%внедрения советов: статический и динамический способы.
%Статический способ внедрения аспектов подразумевает внедрение советов
%непосредственно в программный продукт в процессе построения.
%Данный подход является более быстрым, в отличии от динамического подхода,
%однако, имеет один недостаток: при изменении аспекта необходимо пересобрать весь
%программный продукт.
%Такой подход использован, например, в расширении AspectJ.
%Динамический подход подразумевает связывание аспектов и исполняемого кода
%непосредственно в процессе исполнения программы. Данный подход уступает в
%быстродействии статическому связыванию, однако, лишен необходимости пересборки
%всего проекта в случае изменения аспектов.

%\begin{figure}[htbp]
%\centering
%\includegraphics[width=\textwidth]{how-to-do-the-actual-research}
%\caption{Рекомендации по проведению исследований в рамках диссертации}%
%\label{fig:how-to-do-research}
%\end{figure}


%%%%%%%%%%%%%%%%%%%%%%%%%%%%%%%%%%%%%%%%%%%%%%%%%%%%%%%%%%%%%%%%%%%%%%%%%%%%%%%%
\section{bar}
%%%%%%%%%%%%%%%%%%%%%%%%%%%%%%%%%%%%%%%%%%%%%%%%%%%%%%%%%%%%%%%%%%%%%%%%%%%%%%%%

\blindtext
It is of great importance that you use correct references in your dissertation.
Resent studies show that it can increase the chances of successful defense
by as much as 3,17 percent~\cite{big,small,russian}.

\Blindtext

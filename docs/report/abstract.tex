
\keywords{%
Аспектно-ориентированное программирование,
Язык Kotlin,
Аспекты
}

\abstractcontent{
Данная работа посвящена разработке аспектно-ориентированного подхода для языка Kotlin.
Аспектно-ориентированный подход позволяет компактно описывать и внедрять сквозную функциональность в программы, написанные в стиле объектно-ориентированного программирования.
При этом, АОП-расширения должны предоставлять удобный синтаксис описания сквозной функциональности, а также учитывать особенности языка программирования.
В связи с этим была поставлена задача, разработать АОП-расширение для языка Kotlin, использующее язык аспектов, схожий с синтаксисом языка Kotlin и учитывающее особенности данного языка программирования.

Для реализации АОП-расширения языка Kotlin был разработан язык описания аспектов и разработан подход статического внедрения аспектов на уровне промежуточного представления программы.
Спроектирована архитектура АОП-расширения Kotlin, реализованы парсеры, составляющие модель аспектов по их описанию.
Реализован модуль внедрения аспектов в промежуточное представление программы.
Проведено функциональное тестирование полученного прототипа.

Разработанное АОП-расширение может быть использовано для внедрения сквозной функциональности в программы на языке Kotlin.
Разработанный язык аспектов описан отдельно от приложения и может быть использован при проектировании других АОП-расширений языка Kotlin.

% Магистерская диссертация посвящена разработке подхода, позволяющего внедрять сквозную функциональность в программы на языке Kotlin.
% Для решения данной задачи был разработан синтаксис аспектов с учетом особенностей языка Kotlin, предложен подход внедрения сквозной функциональности путем модификации промежуточного представления программы.
% Для реализации данного подхода был проведен анализ существующих АОП-расширений и рассмотрены способы внедрения аспектов.

% Для демонстрации данного подхода была разработана архитектура программы.
% Разработана грамматика аспектов с учетом языковых особенностей Koltin и парсеры аспектов.
% Разработан алгоритм построения срезов и применения аспектов к программам на языке Kotlin.
% Проведено тестирование разработанной системы.

% Разработанный прототип позволяет использовать аспектно-ориентированный подход при разработке программ на языке Kotlin и эффективно решать ряд таких задач, как, например, протоколирование и аудит.
}

\keywordsen{
  inheritance,
  polymorphism,
  encapsulation
}

\abstractcontenten{
\blindtext
}


\keywords{%
Аспектно-ориентированное программирование,
Язык Kotlin,
Аспектно-ориентированное расширение
}

\abstractcontent{
Работа посвящена разработке аспектно-ориентированного подхода для языка Kotlin.
Аспектно-ориентированный подход позволяет компактно описывать и внедрять сквозную функциональность в программы, написанные в стиле объектно-ориентированного программирования.
Для большинства популярных языков программирования уже существуют подобные расширения.
Язык Kotlin является новым языком программирования, для которого еще не разработано многих необходимых расширений.
В данной работе решается задача разработки АОП-расширения для языка Kotlin, закрывающего один из пробелов.

Для решения указанной задачи был разработан язык описания аспектов и подход статического внедрения аспектов на уровне промежуточного представления программы.
Спроектирована архитектура АОП-расширения Kotlin, реализованы парсеры, составляющие модель аспектов по их описанию.
Реализован модуль внедрения аспектов на уровне промежуточного представления программы.
Проведено функциональное тестирование разработанного прототипа.

Разработанное АОП-расширение может быть использовано для внедрения сквозной функциональности в программы на языке Kotlin.
Разработанный язык аспектов описан отдельно от приложения и может быть использован при проектировании других АОП-расширений языка Kotlin.
}

\keywordsen{
  Aspect-oriented programming,
  Kotlin programming language,
  Aspect-oriented extension
}

\abstractcontenten{
The work is devoted to the development of an aspect-oriented approach for the Kotlin programming language.
Aspect-oriented approach allows compactly describing and implementing cross-cutting functionality in programs, written in object-oriented style.
Such extensions already exist for most popular programming languages.
The Kotlin is a new programming language for which many necessary extensions have not yet been developed.
In this paper, we solve the problem of developing an AOP extension for the Kotlin programming language, which close the gap in this subject area.

To solve this problem we have developed a language for describing aspects and an approach for static weaving of aspects at the intermediate presentation level of the program.
We have designed the architecture of Kotlin's AOP-extension and have developed parsers which create and fill the model of aspects by their description.
We have implemented the module for weaving aspects at the intermediate presentation level of the program.
Functional testing of the developed prototype has been carried out.

The developed AOP extension can be used to introduce cross-cutting functionality into Kotlin programs.
The developed language of the aspects is described separately from the application and can be used in the design of other AOP extensions of the Kotlin language.
}
